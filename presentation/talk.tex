\documentclass{beamer}

\usepackage[utf8]{inputenc}
\usepackage[T1]{fontenc}
\usepackage{lmodern}
\usepackage{stmaryrd}

\usepackage{setspace}
\usepackage{listings}
\lstset{columns=flexible,basicstyle=\ttfamily,keywordstyle=\color{blue}}

\usepackage{tikz}
\usetikzlibrary{positioning}

\newenvironment{reference}{\begingroup\scriptsize}{\endgroup}
\DeclareMathOperator{\pure}{pure}
\newcommand{\ap}{\diamond}
\newcommand{\lba}[1]{\llbracket #1\rrbracket}

\title{Applicative Functors in Isabelle/HOL}
\author{Joshua Schneider \\ \texttt{joshuas@student.ethz.ch}}
%\date{December 8, 2015}

\begin{document}

\begin{frame}
\titlepage
\end{frame}

\begin{frame}
\frametitle{Outline}
\tableofcontents
\end{frame}

\section{Applicative Functors} %%%%%%%%%%%%%%%%%%%%%%%%%%%%%%%%%%%%%%%%%%%%%%%%

\begin{frame}[fragile]
\frametitle{Example: Tree Labels}

\begin{lstlisting}[language=Haskell]
data Tree a = Leaf a | Node (Tree a) (Tree a)
\end{lstlisting}

\begin{center}\begin{tikzpicture}[
	inner/.style={inner sep=0},
	leaf/.style={draw, circle},
	subst/.style={dotted},
	sibling distance=60, level distance=30, on grid]
\node[inner] {}
	child { node[leaf] (n1) {c} }
	child { node[inner] {}
		child { node[leaf] (n2) {a} }
		child { node[leaf] (n3) {b} }
	};
\node[below=2 of n1] {1} edge[subst] (n1);
\node[below=1 of n2] {2} edge[subst] (n2);
\node[below=1 of n3] {3} edge[subst] (n3);
\end{tikzpicture}\end{center}

\vspace{\fill}
\begin{reference}
Inspired by
G.\ Hutton and D.\ Fulger, ``Reasoning About Effects: Seeing the Wood Through the Trees.''
in \emph{Proceedings of the Symposium on Trends in Functional Programming}, (Nijmegen, The Netherlands, 2008).
\end{reference}
\end{frame}

\begin{frame}[fragile]
\frametitle{Composing Stateful Computations}
Standard solution: state monad
\vspace{5mm}
\begin{lstlisting}[language=Haskell]
fresh = do
  x <- get
  put (x + 1)
  return x

numberTree (Leaf _)   = do
  x <- fresh
  return (Leaf x)
numberTree (Node l r) = do
  l' <- numberTree l
  r' <- numberTree r
  return (Node l' r')
\end{lstlisting}
\end{frame}

\begin{frame}[fragile]
\frametitle{Short Circuit Evaluation}

\begin{center}\begin{tikzpicture}[
	inner/.style={inner sep=0},
	leaf/.style={draw, circle},
	subst/.style={dotted},
	sibling distance=60, level distance=30, on grid]
\node[inner] {}
	child { node[leaf] (n1) {c} }
	child { node[inner] {}
		child { node[leaf] (n2) {a} }
		child { node[leaf] (n3) {b} }
	};
\node[below=2 of n1] (e1) {$f(\mathsf{c})$} edge[subst] (n1);
\node[below=1 of n2] (e2) {$f(\mathsf{a})$} edge[subst] (n2);
\node[below=1 of n3] (e3) {$f(\mathsf{b})$} edge[subst] (n3);
\path (e1) -- node {$+$} (e2);
\path (e2) -- node {$+$} (e3);
\end{tikzpicture}\end{center}
$f$ is partial! \hspace{5mm} \lstinline{f :: a -> Maybe Int}

\begin{lstlisting}[language=Haskell]
evalTree (Leaf x)   = f x
evalTree (Node l r) = do
  l' <- evalTree l
  r' <- evalTree r
  return (l' + r')
\end{lstlisting}
\end{frame}

\begin{frame}[fragile]
\frametitle{Cue Applicative Functors}

\begin{lstlisting}[language=Haskell]
class Functor f => Applicative f where
  pure  :: a -> f a
  (<*>) :: f (a -> b) -> f a -> f b
\end{lstlisting}

\begin{lstlisting}[language=Haskell]
numberTree (Leaf _)   = pure Leaf <*> fresh
numberTree (Node l r) =
  pure Node <*> numberTree l <*> numberTree r

evalTree (Leaf x)   = f x
evalTree (Node l r) =
  pure (+) <*> evalTree l <*> evalTree r
\end{lstlisting}
\end{frame}

\begin{frame}
\frametitle{The Laws}
To do: Applicative functor laws \\
To do: Notation $\pure$, $\ap$

\vspace{\fill}
\begin{reference}
C.\ McBride and R.\ Paterson, ``Applicative Programming with Effects.''
\emph{Journal of Functional Programming, 18} (1). 2008, 1--13.
\end{reference}
\end{frame}

\section{Proving Lifted Equations} %%%%%%%%%%%%%%%%%%%%%%%%%%%%%%%%%%%%%%%%%%%%

\begin{frame}
\frametitle{Lifting Terms and Equations}

Lift the term $\operatorname{f} a + b$: \\

\vspace{5mm}
\begin{tabular}{rccrcccc}
$(+)$      &       &(& f      &       & $a$) &       &$b$ \\
pure $(+)$ & $\ap$ &(& pure f & $\ap$ & $a$) & $\ap$ & $b$
\end{tabular}
\\ To do: Idiomatic term

\vspace{10mm}
Lift an equation: Addition is commutative
\[ x + y = y + x \]
\[ \pure{(+)} \ap x \ap y = \pure{(+)} \ap y \ap x \]
\end{frame}

\begin{frame}
\frametitle{Hinze's Lemmas (1)}

\begin{lemma}[Normal Form]
Let $e$ be an idiomatic term with variables $x_1,\dots,x_n$.
There exists $f$ such that
\[ e = \pure f \ap x_1 \ap \dots \ap x_n. \]
\end{lemma}

Can lift equations
\begin{enumerate}
\item where both sides have the same list of variables, and
\item no variable is repeated.
\end{enumerate}

\vspace{\fill}
\begin{reference}
R.\ Hinze, ``Lifting Operators and Laws.'' 2010. Retrieved June 6, 2015,
\url{http://www.cs.ox.ac.uk/ralf.hinze/Lifting.pdf}
\end{reference}
\end{frame}

\begin{frame}
\frametitle{Hinze's Lemmas (2)}

\begin{lemma}[Lifting Lemma]
To do.
\end{lemma}
\end{frame}

\begin{frame}
\frametitle{A Proof Method for Isabelle}

\begin{block}{Project Goal}
Implement a proof method for Isabelle/HOL which lifts equations to
applicative functors.
\[ \text{base equation} \Longrightarrow \text{lifted equation} \]
\end{block}

\vspace{5mm}
Proof method: User interface for goal state transformation
\[ \text{base equation} \longmapsfrom \text{lifted equation} \]
\end{frame}

\begin{frame}
\frametitle{Overview of Operation}

\[ e_1 = e_2 \]
\begin{enumerate}
\item Transform into \emph{canonical forms}
\[ \Longleftarrow\quad \pure f \ap x_1 \ap \dots \ap x_n = \pure g \ap x_1 \ap \dots x_n \]
\item \[ \Longleftarrow\quad f = g \]
\item \[ \Longleftarrow\quad \forall y_1 \dots y_n. \; f y_1 \dots y_n = g y_1 \dots y_n \]
\end{enumerate}
\end{frame}


\section{Spotlight: Combinators} %%%%%%%%%%%%%%%%%%%%%%%%%%%%%%%%%%%%%%%%%%%%%%

\begin{frame}
\frametitle{Combinatory Logic}

\begin{itemize}
\item Eliminate variables from terms, introduce combinator constants
\item BCKW system is equivalent to lambda calculus
\begin{align*}
\mathbf{B} g f x &= g (f x) \\
\mathbf{C} f x y &= f y x \\
\mathbf{K} x y &= x \\
\mathbf{W} f x &= f x x
\end{align*}
\item If not all combinators are available, not all terms can be represented.
\end{itemize}
\end{frame}

\begin{frame}
\frametitle{Bracket Abstraction}

Turn lambda terms into combinator representation

\begin{align*}
\quad& \lambda x y.\> x (f y) \\
=\quad& \lambda x.\> [y](x (f y)) \\
=\quad& \lambda x.\> \mathbf{B} x [y](f y) \\
=\quad& \lambda x.\> \mathbf{B} x f \\
=\quad& [x](\mathbf{B} x f) \\
=\quad& \mathbf{C} [x](\mathbf{B} x) f \\
=\quad& \mathbf{C B} f
\end{align*}

\[ \mathbf{C B} f x y = x (f y) \]
\end{frame}

\begin{frame}
\frametitle{Fancier Idioms}

Some idioms satisfy additional laws, one or more of
\begin{align*}
\pure \mathbf{C} \ap f \ap x \ap y &= f \ap y \ap x \tag{c}\\
\pure \mathbf{K} \ap x \ap y &= x \tag{k}\\
\pure \mathbf{W} \ap f \ap x &= f \ap x \ap x \tag{w}
\end{align*}

Examples
\begin{itemize}
\item sum type $\alpha + \beta$: (w)
\item environment functor $\beta \Rightarrow \alpha$: (c), (k), (w)
\end{itemize}

\vspace{5mm}
Interchange law?
\[ \pure{(\lambda y g.\>g y)} \ap \pure x \ap f = f \ap \pure x \]
\end{frame}

\begin{frame}
\frametitle{Bracket Abstraction Revisited}

Assume an applicative functors satisfies (c).
\begin{alignat*}{2}
\quad& \lambda x y.\> x (f y) & \quad& x \ap (\pure f \ap y) \\
=\quad& \lambda x.\> [y](x (f y)) & \qquad=\quad& \lba{y}(x \ap (\pure f \ap y)) \ap y \\
=\quad& \lambda x.\> \mathbf{B} x [y](f y) & \qquad=\quad& (\mathbf{B} \ap x \ap \lba{y}(\pure f \ap y)) \ap y \\
=\quad& \lambda x.\> \mathbf{B} x f & \qquad=\quad& (\pure \mathbf{B} \ap x \ap \pure f) \ap y \\
=\quad& [x](\mathbf{B} x f) & \qquad=\quad& \lba{x}(\pure \mathbf{B} \ap x \ap \pure f) \ap x \ap y \\
=\quad& \mathbf{C} [x](\mathbf{B} x) f & \qquad=\quad& (\pure \mathbf{C} \ap \lba{x}(\pure \mathbf{B} \ap x) \ap \pure f) \ap x \ap y \\
=\quad& \mathbf{C B} f & \qquad=\quad& \pure \mathbf{C} \ap \pure \mathbf{B} \ap \pure f \ap x \ap y \\
& & \qquad=\quad& \pure{(\mathbf{C B} f)} \ap y \ap x \\
& & \qquad=\quad& \underbrace{\pure{(\lambda x y.\> x (f y))} \ap y \ap x}_\text{a canonical form}
\end{alignat*}
\end{frame}

\begin{frame}
\frametitle{What are the Variables?}

\begin{itemize}
\item Remember that both canonical forms need the same variable lists:
\[ \pure f \ap x_1 \ap \dots \ap x_n = \pure g \ap x_1 \ap \dots x_n \]
\item Must be able to represent terms with available combinators
\item Instantiation:
\begin{alignat*}{2}
	& \forall x y. \; & \pure f \ap x \ap y &= \dots \\
	\Longrightarrow\quad & \forall z. \; & \pure f \ap z \ap z &= \dots
\end{alignat*}
What if we want to prove the latter, but can only represent the former?
\item Algorithm depends on available combinators, partially a heuristic
\end{itemize}
\end{frame}


\section{Demo and Conclusion} %%%%%%%%%%%%%%%%%%%%%%%%%%%%%%%%%%%%%%%%%%%%%%%%%

\begin{frame}
\frametitle{Instances of Applicative}
To do.
\end{frame}

\begin{frame}
\frametitle{To do}
\end{frame}

\end{document}
