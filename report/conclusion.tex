\chapter{Conclusion}\label{sec:conclusion}

\section{Related Work}\label{subsec:related-work}

There already exist libraries for Isabelle/HOL which perform certain kinds of
lifting.
In this section, we want to highlight two of them and discuss how they compare
to our solution.
The first package~\cite{huffman05} implements a ``transfer principle''
specifically designed for nonstandard analysis (NSA).
It focuses on a fixed target type constructor, $\alpha\>\mathit{star}$.
Certain instances of $\alpha\>\mathit{star}$ constitute the nonstandard number
types.
For instance, the hypernatural numbers are elements of $\mathit{nat}\>\mathit{star}$.
Arithmetic operators and their properties have to be lifted in order to develop
their theory, which is automated by the transfer tactic.
Its goal is indeed very similar to ours, but just focused on one application.
Interestingly, the author of the NSA transfer package made use of the applicative
structure of $\alpha\>\mathit{star}$ to define lifted operators, as well as
predicates.
The latter is something the package presented in this report does not support.
We will briefly discuss it as a potential extension in the next section.

The automated lifting of properties is not generic over idioms, though.
This makes it possible to state all required rules directly as polymorphic
HOL theorems.
The tactic takes some lifted proposition, tries to find the base form by
deleting certain constants (compare this to our unfolding approach), and then
shows logical equivalence of the two by backchaining with suitable transfer
rules.
There is only little control by the ML implementation, as the process is mainly
guided by resolution with rule patterns.
Note that all parts of the input proposition go through the same process,
including logical operators such as quantifiers and equality.
This is a good place to mention that $\alpha\>\mathit{star}$ is based
on the function type $\mathit{nat} \funT \alpha$.
Therefore, all combinators can be lifted; variables can be handled by
resolution, because their order or duplication do not matter; all logical
operators are preserved by lifting (the lifted type can be thought of
as a collection of parameterized copies of the base formula).

We observe that the NSA transfer package is a lot more general than ours with
respect to lifted propositions, but is restricted to one particular idiom.
Many of the additional logical operators can only be lifted because of its
special properties.

The other package collection we want to discuss provides quotient types.
A quotient type is created from an existing type, the representation, and an
equivalence relation on it; its elements are isomorphic to the equivalence
classes.
The \emph{Lifting} and \emph{Transfer} packages~\cite{huffman13} constitute a
modular infrastructure for transferring results from the representation level
to the new type.
\emph{Transfer} finds and proves equivalences or implications between
propositions according to an extensible set of transfer rules.
These describe how two different functions (or type instances of the same
function) are related, given relations of their arguments.
Transfer is heavily driven by the type structure, following the principle of
relational parametricity.
The goal of \emph{Lifting} is to automate the specification of constants lifted
to quotient types.
Because this kind of lifting preserves properties by construction, these should
be carried over as well.
Therefore, the package uses transfer to relate the new constant with its
representation.

The relations used by transfer cannot express lifting to applicative functors.
Values of the latter are usually more complex than those of the base type,
instead of more abstract.
We would need to find relations which are strong enough to lift arbitrary
predicates.
This is clearly not possible for functors with effects.
To summarize, the actual lifting step of our package is very simple, because it
just considers pairs of extensionally equal functions;
the whole automation is necessary to extract these functions from idiomatic
expressions in the first place.
Transfer, on the other hand, preserves the term structure, but relates each
component of the term, and composes these relations according to the
transfer rules.


\section{Summary and Future Extensions}\label{subsec:summary-future}

We have successfully implemented Hinze's applicative lifting as a proof method
for Isabelle/HOL.
It can be instantiated for arbitrary idioms, making it a (hopefully) reusable
tool.
Since it supports several combinator sets, we take advantage of the individual
properties of the idioms.
The proof methods has been used to produce direct proofs of various algebraic
laws for streams and infinite binary trees.

The choice of a shallow embedding led to a rather compact implementation,
next to the registration infrastructure, which is necessary either way.
The highly polymorphic situation demands constant care for type parameter
instantations, which is tedious.
On the other hand, this is a problem for a deep embedding as well.

The current implementation is restricted to equations.
As we have seen above, certain idioms can lift quite a lot more.
A practical example are cancellation laws of a lifted binary operator, which
have the structure $\_ = \_ \implies \_ = \_$ (see also
Section~\ref{subsec:lifted-algebra}).
We had to prove these manually for the stream and infinite tree theories.
Again, variables make our live difficult:
The lifted assumption does not necessarily imply the base assumption for
\emph{all} instances, and we cannot eliminate the implication to continue
with the conclusion only.
It is not obvious which additional conditions are necessary or sufficient to
lift such properties.
We have seen that the ``power'' of the various idioms differs with respect to
combinators.
Those which are homomorphic to the environment functor (streams, etc.) seem to
be the most versatile.
It might be interesting to survey possible categorizations and how they lead
to further proof automation.
