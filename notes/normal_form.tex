\section{Normal Form Conversion}\label{sec:normal-form}

McBride and Paterson~\cite{mcbride08} noted that idiomatic expressions can
be transformed into an application of a pure function to a sequence of impure
arguments.
Hinze~\cite{hinze10} gave an explicit construction of this normal form for the
monoidal variant of applicative functors.
The normal form is useful for our purpose because the pure part reflects the
term that was lifted.
Its construction can be performed using only the applicative laws, thus this is
the most general approach regarding instances (but not regarding lifted
equalities).
In the following, we define lifting and normalization formally, based on a
syntactic representation of idiomatic terms.
Then we describe the implementation of the normalization procedure in Isabelle.

\subsection{The Idiomatic Calculus}\label{subsec:idiomatic-calculus}

Though we are mainly interested in idiomatic terms, we need to represent
pure and (opaque) impure values.
It makes sense to refer to general lambda terms;
then we can define the semantics consistently.
However, types are ignored for simplicity.

\begin{definition}[Untyped lambda terms]
Let $\mathcal{V}$ be an infinite set of variable symbols.
We assume that $f$, $g$, $x$, $y$ are disjoint variables.
The set of untyped lambda terms is defined as
\begin{equation}
	\mathcal{T} ::= \mathcal{V} \mid (\mathcal{T} \sapp \mathcal{T}) \mid
		\sabs{\mathcal{V}}{\mathcal{T}}
\end{equation}
An equivalence relation on $\mathcal{T}$ is a $\mathcal{T}$-congruence iff it
is closed under application and abstraction.
Let $=_\beta$ be the smallest $\mathcal{T}$-congruence containing $\alpha$- and
$\beta$-conversion.
\end{definition}

\begin{definition}[Idiomatic terms]
The set of idiomatic terms is defined as
\begin{equation}
	\mathcal{I} ::= \simp \mathcal{T} \mid \spure \mathcal{T} \mid
		\mathcal{I} \sap \mathcal{I}.
\end{equation}
An $\mathcal{I}$-congruence is an equivalence relation closed under $\sap$.
The congruence $\simeq$ is induced by the rules
\begin{align}
	\spure{(\sabs{x}{x})} \sap x &\simeq x \label{eq:term-id}\\
	((\spure \mathbb{B} \sap g) \sap f) \sap x &\simeq g \sap (f \sap x) \label{eq:term-comp}\\
	\spure f \sap \spure x &\simeq \spure (f \sapp x) \label{eq:term-morph}\\
	f \sap \spure x &\simeq \spure{((\sabs{x}\sabs{f}{f \sapp x}) \sapp x)} \sap f \label{eq:term-xchng}
\end{align}
where $\mathbb{B}$ abbreviates
$\mathbb{B} = \sabs{g}{\sabs{f}{\sabs{x}{g \sapp (f \sapp x)}}}$.
\end{definition}

\begin{definition}[Normal form]
The set $\mathcal{N} \subset \mathcal{I}$ of idiomatic terms in normal form is
defined inductively as
\begin{gather}
	\spure x \in \mathcal{N}, \label{eq:nf-base}\\
	t \in \mathcal{N} \implies t \sap \simp x \in \mathcal{N}. \label{eq:nf-step}
\end{gather}
\end{definition}

\begin{algorithm}
\caption{Normalization of idiomatic terms.}
\label{alg:normalize}
\begin{align}
	\operatorname{pure-nf} t &= \begin{cases}
		\operatorname{pure-nf}{(\spure{(\mathbb{B} g)} \sap f)} \sap x
			&\text{if } t = \spure g \sap (f \sap x), \\
		\spure{(f x)} &\text{if } t = \spure f \sap \spure x, \\
		t &\text{otherwise.}
	\end{cases} \label{eq:pure-nf}\\
	\operatorname{nf-pure} t &= \begin{cases}
		\operatorname{pure-nf}{(\spure{((\sabs{x}\sabs{f}{f x}) \sapp x)} \sap f)}
			&\text{if } t = f \sap \spure x, \\
		t &\text{otherwise.}
	\end{cases} \label{eq:nf-pure}\\
	\operatorname{nf-nf} t &= \begin{cases}
		\operatorname{nf-nf}{(\operatorname{pure-nf}{(\spure \mathbb{B} \sap g)} \sap f)} \sap x
			&\text{if } t = g \sap (f \sap x), \\
		\operatorname{nf-pure} t &\text{otherwise.}
	\end{cases} \label{eq:nf-nf}\\
	\operatorname{normalize} t &= \begin{cases}
		t &\text{if } t = \spure x, \\
		\spure{(\sabs{x}{x})} \sap t &\text{if } t = \simp x, \\
		\operatorname{nf-nf}{(\operatorname{normalize} x \sap \operatorname{normalize} y)}
			&\text{if } t = x \sap y.
	\end{cases} \label{eq:normalize-def}
\end{align}
\end{algorithm}

\begin{lemma}\label{thm:normalize-alg}
$\operatorname{normalize}$ (cf.\ algorithm~\ref{alg:normalize}) terminates for all
inputs $t \in \mathcal{I}$, and%
\footnote{$a \in S \simeq b$ abbreviates ``$a \in S$ and $a \simeq b$''.}
$\operatorname{normalize} t \in \mathcal{N} \simeq t$.
\end{lemma}
\begin{proof}
The proof proceeds bottom-up, proving results about the auxiliary definitions
first.
pure-nf terminates because the number of $\sap$ constructors in the argument
of the recursive call decreases strictly.
We show that for all $f \in \mathcal{T}$ and $x \in \mathcal{N}$,
$\operatorname{pure-nf}(\spure f \sap x) \in \mathcal{N} \simeq \spure f \sap x$;
by induction on $x$ via $\mathcal{N}$, where $f$ is arbitrary:
\begin{prfcases}
\item $x = \spure x'$ for some $x' \in \mathcal{T}$. Then,
	\[ \operatorname{pure-nf}{(\spure f \sap \spure x')} = \spure{(f \sapp x')} \in \mathcal{N}. \]
	Equivalence holds by \eqref{eq:term-morph}.
\item $x = x' \sap \simp y$ for some $x' \in \mathcal{N}$, $y \in \mathcal{T}$.
	The induction hypothesis is
	\[ \operatorname{pure-nf}{(\spure f' \sap x')} \in \mathcal{N} \simeq \spure f' \sap x' \]
	for all $f' \in \mathcal{T}$.
	By definition, we have
	\[ \operatorname{pure-nf}{(\spure f \sap (x' \sap \simp y))} =
		\operatorname{pure-nf}{(\spure{(\mathbb{B} \sapp f)} \sap x')} \sap \simp y. \]
	Instantiating the hypothesis with $f' = \mathbb{B} \sapp f$ and using \eqref{eq:nf-step}
	it follows that
	\[ \operatorname{pure-nf}{(\spure f \sap (x' \sap \simp y))} \in \mathcal{N} \simeq
		(\spure{(\mathbb{B} \sapp f)} \sap x') \sap \simp y. \]
	It remains to show that the right hand side is equivalent to $\spure f \sap x$:
	\begin{align*}
		\spure f \sap x &= \spure f \sap (x' \sap \simp y) \\
		&\stackrel{\mathclap{\eqref{eq:term-comp}}}{\simeq} ((\spure \mathbb{B} \sap \spure f) \sap x') \sap \simp y \\
		&\stackrel{\mathclap{\eqref{eq:term-morph}}}{\simeq} (\spure{(\mathbb{B} \sapp f)} \sap x') \sap \simp y.
	\end{align*}
\end{prfcases}

Termination of nf-pure is trivial.
Consider arbitrary but fixed $f \in \mathcal{N}$ and $x \in \mathcal{T}$.
From the previous result,
\[ \operatorname{pure-nf}(\spure{((\sabs{x}\sabs{f}{f \sapp x}) \sapp x)} \sap f)
	\in \mathcal{N} \simeq \spure{((\sabs{x}\sabs{f}{f \sapp x}) \sapp x)} \sap f, \]
hence clearly $\operatorname{nf-pure}(f \sap \spure x) \in \mathcal{N}$.
Equivalence follows from \eqref{eq:term-xchng}.

Finding a termination measure for nf-nf is more involved.
It turns out that we can use the number of leaves ($\simp$ and $\spure$ instances)
in an idiomatic term, where the left-most $\spure$ is ignored.
More formally, we define functions $m,m' : \mathcal{I} \to \mathrm{Nat}$
\begin{align*}
	m(\spure x) &= 1, & m(\simp x) &= 1, & m(f \sap x) &= m(f) + m(x); \\
	m'(\spure x) &= 0, & m'(\simp x) &= 1, & m'(f \sap x) &= m'(f) + m(x).
\end{align*}
The first observation is that
\[ \all{f x}{m'(\operatorname{pure-nf}(\spure f \sap x)) \leq m'(x)}. \]
The simple proof by induction is omitted.
Because $m(x) > 0$ for all $x \in \mathcal{I}$, we have
\begin{align*}
	&\mathrel{\phantom{=}} m'(\operatorname{pure-nf}(\spure \mathbb{B} \sap g) \sap f) \\
	&= m'(\operatorname{pure-nf}(\spure \mathbb{B} \sap g)) + m(f) \\
	&\leq m'(g) + m(f) \\
	&< m'(g) + m(f) + m(x) = m'(g \sap (f \sap x)).
\end{align*}
This relates the argument to the recursive invocation of nf-nf with the
original argument.
Since $<$ is a well-founded relation on the natural numbers, the termination
proof is complete.

We now show that for all $f,x \in \mathcal{N}$,
$\operatorname{nf-nf}(f \sap x) \in \mathcal{N} \simeq f \sap x$.
\todo

Termination of `normalize' is once again due to the number of $\sap$ constructors.
Finally, it remains to show that for all $t \in \mathcal{I}$,
$\operatorname{normalize} t \in \mathcal{N} \simeq t$.
This time, induction on the term structure of $t$ is used:
\begin{prfcases}
\item $t = \spure x$ for some $x \in \mathcal{T}$.
	This case is trivial:
	\[ \operatorname{normalize}(\spure x) = \spure x \in \mathcal{N} \simeq \spure x. \]
\item $t = \simp y$ for some $y \in \mathcal{T}$.
	Then
	\[ \operatorname{normalize}(\simp y) = \spure{(\sabs{x}{x})} \sap \simp y \in \mathcal{N} \]
	by \eqref{eq:nf-base} and a single step of \eqref{eq:nf-step}.
	Equivalence follows from \eqref{eq:term-id}.
\item $t = x \sap y$ for some $x,y \in \mathcal{I}$.
	The induction hypothesis is
	$\operatorname{normalize}(x) \in \mathcal{N} \simeq x$
	and similarly for $y$.
	We then use the result for nf-nf and obtain
	\[ \operatorname{nf-nf}(\operatorname{normalize} x \sap \operatorname{normalize} y)
		\in \mathcal{N} \simeq x \sap y. \qedhere \]
\end{prfcases}
\end{proof}

\begin{lemma}
For all $t \in \mathcal{I}$, there is a unique term $t' \in \mathcal{N}$ such
that $t \simeq t'$.
\end{lemma}
\begin{proof}
Existence of the normal form is a corollary of Lemma~\ref{thm:normalize-alg}.
\todo\ prove uniqueness (if it is actually true)
\end{proof}

Until now, we only have considered the syntactic structure of idiomatic terms
together with the artificial relation $\simeq$, which is also based on syntax.
In order to define the semantics of idiomatic terms, we assume that we operate
in an equational theory $\Omega$ based on $\mathcal{T}$-terms, where
$=_\Omega \supseteq =_\beta$ is an equivalence relation.

\begin{definition}[Idiomatic interpretation]
Let $\iota = \langle p, a \rangle$ with $p,a \in \mathcal{T}$.
The interpretation $\ldb t \rdb_\iota$ of the idiomatic term $t$ w.r.t. $\iota$
is defined as
\begin{align}
	\ldb \simp t \rdb_\iota &= t, \\
	\ldb \spure t \rdb_\iota &= p \sapp t, \\
	\ldb s \sap t \rdb_\iota &= (a \sapp \ldb s \rdb_\iota) \sapp \ldb t \rdb_\iota.
\end{align}
$\iota$ is an idiomatic structure (in $\Omega$) iff
\begin{equation}
	\all{q r}{q \simeq r \implies \ldb q \rdb_\iota =_\Omega \ldb r \rdb_\iota}.
\end{equation}
\end{definition}

\begin{definition}
\begin{equation}
	\iota_\mathrm{id} = \langle \sabs{x}{x}, \sabs{f}{\sabs{x}{f \sapp x}} \rangle
\end{equation}
\end{definition}

\begin{lemma}
\todo\ Every encoding of an applicative functor in $\Omega$ is an idiomatic
structure.
Especially $\iota_\mathrm{id}$ is an idiomatic structure.
\end{lemma}

\begin{definition}[Lifted terms]
$q \in \mathcal{I}$ is a lifting of $t \in \mathcal{T}$ if 
$\ldb q \rdb_{\iota_\mathrm{id}} =_\Omega t$.
\end{definition}

\begin{lemma}
Let $q$ be a lifting of $t$. The normal form $q'$ of $q$ can be written as
\[ q' = ( \cdots ((\spure t' \sap \simp a_1) \sap \simp a_2) \cdots ) \sap \simp a_n. \]
Then $t' \sapp \vec a =_\Omega t$.
\end{lemma}
\todo
