\section{Project Overview}\label{sec:overview}

\subsection{Introduction}\label{subsec:introduction}

% TODO moved
Our primary goal is to implement an Isabelle/HOL proof method which reduces
lifted equations to their base form.
Here, lifting refers to a transition from operations on base types to related
operations on some structure.
Hinze~\cite{hinze10} studied the conditions under which lifting preserves the
validity of equations.
He noticed that lifting can be defined in an intuitive fashion if the target
structure is an applicative functor~\cite{mcbride08}:
a unary type constructor $f$ with associated constants%
\footnote{Types are given in Isabelle notation.}
\begin{align*}
	\pure_f &\oftype \alpha \funT \alpha f, \\
	(\ap_f) &\oftype (\alpha \funT \beta) f \funT \alpha f \funT \beta f.
\end{align*}
The operator $\ap_f$ is left-associative.
We omit the subscripts if the functor is clear from the context.
Moreover, the following laws must be satisfied:
\begin{align*}
	\tag{identity} \pure{\mathit{id}} \ap u &= u \\
	\tag{composition} \pure{(\cdot)} \ap u \ap v \ap w &= u \ap (v \ap w) \\
	\tag{homomorphism} \pure{f} \ap \pure x &= \pure{(f x)} \\
	\tag{interchange} u \ap \pure{x} &= \pure{(\abs{f}{f x})} \ap u
\end{align*}

The identity type constructor defined by $\alpha\,\mathit{id} = \alpha$ is a
trivial applicative functor for $\pure{x} = x$, $f \ap x = f x$.
We can take any abstraction-free term $t$ and replace each constant $c$ by
$\pure{c}$, and each instance of function application $f x$ by $f \ap x$.
The rewritten term is equivalent to $t$ under the identity functor
interpretation, or identity ``idiom'' as coined in \cite{mcbride08}.
By choosing a different applicative functor, we obtain a different
interpretation of the same term structure.
In fact, this is how we define the lifting of $t$ to an idiom.
We also permit variables, which remain as such in the lifted term, but range
over the structure instead.
A term consisting only of $\pure$ and $\ap$ applications and
free variables is called an idiomatic expression.

\begin{example}\label{exmp:set-intro}
Another applicative functor can be constructed from sets.
For each type $\alpha$ there is a corresponding type $\alpha\,\mathit{set}$
of sets with elements in $\alpha$;
$\pure$ denotes the singleton set constructor $x \mapsto \{x\}$;
$F \ap X$ takes a set of functions $F$ and a set of arguments $X$
with compatible type, applying each function to each argument:
\[ F \ap X = \set{f x}{f \in F,\, x \in X}. \]
We can lift addition on natural numbers to the set idiom by defining the operator
\begin{align*}
	(\oplus) &\oftype \mathit{nat}\,\mathit{set} \funT \mathit{nat}\,\mathit{set} \funT
		\mathit{nat}\,\mathit{set}, \\
	X \oplus Y &= \pure{(+)} \ap X \ap Y = \set{x + y}{x \in X,\, y \in Y}.
\end{align*}
The associative property of addition
\[ \all{x y z}{(x + y) + z = x + (y + z)} \]
can be translated to sets of natural numbers
\[ \all{X Y Z}{(X \oplus Y) \oplus Z = X \oplus (Y \oplus Z)}, \]
where it holds as well, as one can check with a slightly laborious proof.
Note that the two sides of the latter equation are the lifted counterparts
of the former, respectively.
\end{example}

As we have seen, lifting can be generalized to equations.
There is actually a more fundamental relationship between the two equations
from above example---the lifted form can be proven for all applicative
functors, not just $\mathit{set}$, using only the base property and the
applicative functor laws.
We want to automate this step with a proof method.

Not all equations can be lifted in all idioms, though.
Stronger conditions are required if the list of quantified variables is
different for each side of the equation.
(The left-to-right order is relevant, but not the nesting within the terms.)
These conditions must basically ensure that the functor does not add ``too many
effects'' which go beyond the simple embedding of a base type.
Such effects may be evoked if a variable takes an impure value, i.e., a value
which is not equal to $\pure x$ for any $x$.

\begin{example}\label{exmp:set-counterexmp}
We try to lift the fact that zero is a left absorbing element for
multiplication of integers, $\all{x \oftype \mathit{int}}{0 \cdot x = 0}$,
to sets.
Note that the variable $x$ occurs only on the left.
But the lifted equation does not hold: If $x$, now generalized to
$\mathit{int}\,\mathit{set}$, is instantiated with the empty set, then
\[ \pure{(\cdot)} \ap \pure 0 \ap \{\} = \{\} \ne \pure 0. \]
Here the effect of $\{\}$ is that it cancels out everything else if it occurs
somewhere in an idiomatic expression.
This makes it impossible to lift any equation with a variable occuring only on
one side to $\mathit{set}$.
\end{example}
