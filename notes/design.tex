\section{Requirements and Basic Design}\label{sec:design}

\subsection{Requirements}

Our primary goal is to implement an Isabelle/HOL proof method which reduces
lifted equations to their base form.
This proof method should be generic and work with arbitrary idioms.
The following is the minimal set of user actions we shall support:

\begin{enumerate}
\item Declare applicative functors to the theory context.
	Given a type constructor $f$, the functions $\pure_f$ and $\ap_f$, and
	proofs for the relevant functor laws, the functor is registered with the
	package such that it can be used in subsequent invocations of the
	proof method.
\item Prove lifted equations $a'[\vec{x}] = b'[\vec{x}]$, where $a'$ and $b'$
	are idiomatic expressions with free variables $\vec{x}$, using the base
	equation $\all{\vec{y}}{a[\vec{y}] = b[\vec{y}]}$.
	More precisely, if there is a subgoal stating the former, applying the
	proof methods transforms the goal state to the latter.
	The functor should be either detected automatically, or specified by
	the user.
\end{enumerate}

The first requirement ensures that our package is reusable, while the second
is the core functionality.
However, usability is also a concern: In the realm of interactive theorem
proving, it is not sufficient to just verify formal objects---we are not
extending the logic, after all, but providing a shortcut for a certain
principle.
We must balance the clarity of the resulting proof document and the amount of
work that the user has to put into developing a proof.
The following features are possible extensions which may help in this regard:

\begin{enumerate}
\setcounter{enumi}{2}
\item\label{itm:feat-const}
	Declare lifted constants and other terms to the theory context.
	Generally speaking, if a term $t$ with free variables $v_i$ can be expressed
	as $\pure t' \ap v_1 \ap \cdots \ap v_n$ for some $t'$, then the
	corresponding equation can be registered.
	The lifting proof methods rewrites with these equations at the beginning,
	such that the base equation will refer to $t'$.
	This way, the user does not have to transform everything into idiomatic
	format first.
\item\label{itm:feat-flex}
	More flexibility regarding the logical structure of the input proposition.
	This includes bound variables (quantified by $\forall$ or $\bigwedge$),
	complex subgoals with premises, and cases where the conceptual variables
	of the lifted equation have been subsituted by some terms.
\item\label{itm:feat-tools}
	Related proof methods and attributes, for example for forward lifting
	of proven base equations.
\item\label{itm:feat-debug}
	Inspection and tracing output.
	This is particular useful if something does not work as expected.
\item\label{itm:feat-xprops}
	Extend the notion of lifting beyond equations.
	It is possible to define lifting for other logical operators.
	For example, the cancellation law $a + b = a + c \longrightarrow b = c$
	consists of two equations, joined by implication.
	We can interpret it in different domains, e.g., for integers and for
	streams of integers with element-wise addition.
	In this example, the law is true for both interpretations.
	We are not able to handle such propositions with just a method for
	equations, though.
	% TODO example may fit better somewhere else?
\end{enumerate}

The current version at the time writing supports \ref{itm:feat-const} and
\ref{itm:feat-flex}, as well as \ref{itm:feat-tools} and \ref{itm:feat-debug}
to some extent.
\todo\ \ref{itm:feat-xprops} not discussed in detail, but see final section. % TODO
In the remainder of this section, we will explain the design decisions we
have made in order to fulfill the requirements.
Furthermore, the registration infrastructure and the basic proof approach are
explained.


\subsection{Choice of Embedding}\label{subsec:embedding}

The basic way of proof composition in Isabelle is resolution with a proven
theorem, using it as a rule of inference.
Here we argue that it is not possible to prove lifting as a single theorem, and
then we discuss potential remedies.
We are interested in applicative functors on the type level, where application
is based on the standard function space.
Therefore, each idiom comes with a type family which is indexed by a
distinguished type variable, and the related functions and laws are polymorphic
in this index.
This is the natural form of idioms in the HOL libraries; all examples in \todo\
are parametric datatypes.  % TODO ref
The proof method should work directly with equations of these idioms.
Regardless of the mechanism of proof construction, it needs a type constructor
as a parameter.
This concept is foreign to the type system of Pure and HOL---%
we already referred to the fact that type constructors are fixed.
Another issue is the lack of polymorphism in the inner logic:
We cannot have, say, a schematic variable \textit{?pure} and use it with
different types within the same proposition or proof.

One solution is to define a custom logic, including a term language, axioms
and meta theorems, and formalize it using the available specification tools.
In our case, the language is that of idiomatic expressions,%
\footnote{Base equations can be interpreted as expressions of the identity idiom.}
the axioms describe an equality judgment which is compatible with the
applicative functor laws, and lifting of equations is a theorem.
This is a \emph{deep embedding} of the logic~\cite{wildmoser04}.
With its tools for algebraic datatypes and recursive functions, reasoning about
such an embedding is quite manageable in HOL.
However, we want to prove propositions involving objects of HOL itself, not just
their encodings in the embedded language.
Some machinery, known as reflection, needs to perform the encoding and transfer
back results.
It must be implemented necessarily outside of the logic, but can be generic.
Chaieb and Nipkow have implemented a proof procedure using a deep embedding and
reflection in Isabelle~\cite{chaieb05}.
They point out that their approach also functions as a verification of the
proof procedure, is portable and has smaller proofs than those obtained by
automating inference rules.
Their reflection system does not support polymorphism to the extent we need,
though.

The package for nonfree datatypes~\cite{schropp13} is deeply embedded as well.
Its constructions must work with arbitrary types.
In the underlying framework, Schropp~\cite{schropp12} proposed the use of a
``pseudo-universe'', a sum type combining all these types.
The meta-theory of the package carries a type parameter which is instantiated
with a suitable pseudo-universe for every construction.
It may be possible to use the same approach for idiomatic expressions, since the
number of types occuring in an idiomatic expression is finite.
This number can be linear in the size of the expression, though, which bloats
the intermediate terms during reflection.
A bigger problem is the generic axiomatization of idioms.
For instance, the identity law would refer to a function of type $\alpha \funT \alpha$
for each type $\alpha$ in the pseudo-universe.
Thus, the universe needs to be closed under function types.
It is not clear to the author how this could be modelled.

A different approach, which is the one we will take, is a \emph{shallow embedding}.
The formulas (here, idiomatic terms) are expressed directly in the language of HOL.
Due to aforementioned restrictions, meta-theorectical results must be provided
in specialized form for each case.
We use the powerful ML interface of Isabelle to program the proof construction,
composing inferences according to the structure of the input equation.
The handling of polymorphism is simplified, as we have full control over
term and theorem instantiations.
The system still verifies the soundness of the synthesized proofs.
On the other hand, one has to assert externally that the construction algorithm
itself is correct, i.e., complete.
The main part of this report therefore justifies these algorithms.


\subsection{User Interface}\label{subsec:interface}

We provide a command to declare applicative functors.
Its syntax is
\begin{isabelle}
	\textbf{applicative} \textit{name} (\textit{combinator}, \dots) \\
	\textbf{for} \\
	\iindent pure: "$\pure_f$" \\
	\iindent ap: "$\ap_f$" \\
	\textit{proof}
\end{isabelle}
The idiom will be made available under the \textit{name}.
It can be used to refer to the idiom manually in proofs.
The name is followed by an optional list of a subset of the symbols C, K, and W.
These declare additional properties which are described in Section~\ref{sec:combinators}.
The functions $\pure$ and $\ap$ imply the type scheme, see also the next
section.
Finally, the idiom laws need to be proven.
The system presents the user with corresponding goals, which are solved by the
proof.
After the command has been issued, the functor can be used in subsequent
commands.
The data is stored in the theory context and thus is imported along other
theory data.
This allows the construction of a reusable idiom library.

Lifted constants may be registered with the attribute \emph{applicative\_unfold},
which can be applied to facts $\mathit{lhs} = \mathit{rhs}$, where $\mathit{rhs}$
is an idiomatic expression.
The equation must be suitable for rewriting.

% TODO below
The complete set of subgoal forms to support has not been determined yet.
\todo{}
As a minimal requirement, after unfolding lifted constants, HOL equations of
idiomatic expressions shall be handled.
Only the outermost functor $f$ is considered per invocation.
The conceptual variables of the lifted expressions may be instantiated with
arbitrary terms.
However, the method actually proves the fully universally quantified form---%
for every subterm not matching $\pure_f{\_}$ or $\_ \ap_f \_$, a new, locally
quantified variable is introduced.
We call these subterms \emph{opaque}.
The method attempts to transform the first subgoal to the base form of the
equation, in other words, its identity functor interpretation.
Variable names shall be preserved, if possible.
Finally, it is desirable to have some kind of debugging facility for tracing
intermediate steps.
This may be especially useful if the proof fails because additional laws are
required.

\begin{example}\label{exmp:set-usage}
Continuing with the set idiom from Example~\ref{exmp:set-intro}, assume that
the user wants to prove an instantiation of the associativity law for $\oplus$
as part of a larger proof.
$\mathit{set}$ and $\oplus$ have been declared to the enclosing theory.
\begin{isabelle}
	\textbf{definition} [applicative\_unfold] "$X \oplus Y = \pure (+) \ap X \ap Y$"
\end{isabelle}
The current Isabelle goal state is
\[ 1. \quad (X \oplus Y) \oplus f Z = X \oplus (Y \oplus f Z). \]
The variables $X$, $Y$ and $Z$ have been fixed in the proof context, and $f$ is
some constant, all of appropriate type.
This illustrates how we allow a larger variety of propositions, making it
easier to apply the method without too much preparation.
After applying the proof method, the new proof obligation reads
\[ 1. \quad \All{x y u}{(x + y) + u = x + (y + u)}, \]
which is easily discharged.
The corresponding Isabelle/Isar fragment could be
\begin{isabelle}
	\textbf{fix} $X$ $Y$ $Z$ \\
	\dots \\
	\textbf{have} "$(X \oplus Y) \oplus f Z = X \oplus (Y \oplus f Z)$"
		\textbf{by} \textit{applicative\_lifting} \textit{algebra}
\end{isabelle}
\textit{applicative\_lifting} is our new proof method, and the standard
\textit{algebra} method completes the subproof (it solves the base equation).
\end{example}


\subsection{Proof Strategy}\label{subsec:proof-strategy}

The ML code of the package can be considered in two parts.
One is concerned with registration of idioms and access to the recorded
information, the other provides the actual proof methods.
We start with a summary of the former component.
It does not only store the idioms declared with the \textbf{applicative}
command, but also deals with the high degree of polymorphism.
First, we need to represent the concept of an idiom.
Instead of a plain type constructor we use type schemes $f[\tvar{a}]$, which
are simply normal HOL types with (in this case) one distinguished type variable
$\tvar{a}$.
This variable has to be tracked externally in ML because of the lack of
type quantifiers in the type system.
The lifted type of $\alpha$ is then obtained by substitution of $\tvar{a}$ by
$\alpha$ in $f[\tvar{a}]$.
The functions $\pure_f$ and $\ap_f$ thus have types
$\tvar{a} \funT f[\tvar{a}]$ and $f[\tvar{a} \funT \tvar{b}] \funT f[\tvar{a}] \funT f[\tvar{b}]$,
respectively.
We allow additional type variables in functor signatures, which therefore
represent families of functors.
These are always instantiated with the same types throughout a proof.
This is useful for idioms like the sum type $\alpha + \beta$.
As a further refinement, all type variables may be constrained by a sort.
Sorts in Pure are type classes~\cite{implementation-ref}.
The sort of the distinguished variable in $f$ must be compatible with the
function type constructor.

Above parts form the signature of the idiom $f$.
Using it, we define functions which compose and destruct types and terms
related to $f$.
One problem is identifying terms of the form $\pure \_$ and $\_ \ap \_$ in
the first place.
We do not want to restrict the term structure of the functor ``constants'',
since some useful predefined concepts in the HOL libraries are abbreviations of
compound terms.
Isabelle's higher-order matching appears to be a proper solution in practice,
using $\pure v$ (or $v_1 \ap v_2$) with new variables $v$ ($v_1$, $v_2$) as the
pattern.

The remaining code uses these functions to provide several layers of
conversions and tactics.
We start at the top.
A wrapper tactic prepares the subgoal to solve.
This includes dealing with universal quantifiers $\forall$ and local premises.
We then rewrite the subgoal using the declared rules for lifted constants.
Only those related to $f$ are used, the reason being that overeager, unwanted
unfolding may be difficult to reverse.
The result of this preparation is a subgoal which is a simple equation of two
idiomatic expressions.
We implement two approaches for the core step.
The tactics have in common that both expressions are replaced by others in
\emph{canonical form}:
\[ \pure{g} \ap s_1 \ap \cdots \ap s_m = \pure{h} \ap t_1 \ap \cdots \ap t_n, \]

Hinze's Normal Form Lemma~\cite[7]{hinze10} asserts the existence of a certain
normal form for idiomatic expressions where each variable occurs only once.
This normal form has the desired structure.
As it turns out, we can compute it for arbitrary terms.
This is convenient because opaque parts are handled implicitly.
However, the result might be too general and unprovable, so we will use the
normal form only if no other properties are available.
The details of the normalization algorithm are described in
Section~\ref{sec:normal-form}.
There we will also show that the transformed equation is exactly the generalized
base form of the original equation.

Hinze then explored under what circumstances a larger variety of equations can
be lifted.
He found that sufficient conditions can be expressed in terms of combinators
as known from combinatory logic.
We take this idea and adapt it our framework in Section~\ref{sec:combinators}.
In contrast to the normal form approach, we gain flexibility regarding the
opaque terms in the canonical form.
This means that the algorithm must determine the sequence $\vec t = \vec s$
of opaque terms prior to the transformations.
The set of available combinators further limits the admissible sequences.
Because this relationship cannot be generalized easily, we restrict ourselves
to certain combinator sets.

After any of the two, we apply appropriate congruence rules until the subgoal is
reduced to $g = h$.
If either $m \ne n$ or $t_i \ne s_i$ for some $i$ (as terms modulo
$\alpha\beta\eta$-conversion), the proof method fails.
Since $g$ and $h$ are at least $n$-ary functions, we can further apply
extensionality, reaching the subgoal
\[ \All{x_1 \dots x_n}{g x_1 \cdots x_n = h x_1 \cdots x_n}. \]
This is the transformed proof state presented to the user.
