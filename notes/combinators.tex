\section{Lifting with Combinators}\label{sec:combinators}

\subsection{Motivation}\label{subsec:combinator-motivation}

The normalization approach to solving lifted equations works only if the
opaque terms on both sides coincide.
This is not true for all equations of interest.
Let's revisit the set version of addition of natural numbers, $\oplus$ from
Example~\ref{exmp:set-intro}.
This operator is also commutative, so it should be possible to prove
\[ X \oplus Y = Y \oplus X. \]
After unfolding and normalization, we get
\begin{equation}\label{eq:comb-intro1}
	\pure{(\abs{xy}{x + y})} \ap X \ap Y = \pure{(\abs{yx}{y + x})} \ap Y \ap X.
\end{equation}
Clearly, this can't be solved with a standard congruence rule, because we would
have to to prove that $X$ is equal to $Y$.
Since we are concerned with transferring properties from a base domain,
we don't want to assume anything about those opaque subterms, which may
carry additional information of the functor.
Note that the arguments of both $\pure$ terms are actually the same function,
so we can't even make use of the base equation there.
Expressed with functions, it reads
\[ \abs{xy}{x + y} = \abs{xy}{y + x}. \]
With the flip function, defined as $\operatorname{flip} fxy = fyx$,
we can however derive
\[ \abs{xy}{x + y} = \operatorname{flip}{(\abs{yx}{y + x})} \]
and further
\[ \pure{(\abs{xy}{x + y})} \ap X \ap Y =
	\pure{(\operatorname{flip}{(\abs{yx}{y + x})})} \ap X \ap Y. \]
Now it would be very convenient if
\begin{equation}\label{eq:comb-intro2}
	\pure \operatorname{flip} \ap f \ap x \ap y = f \ap y \ap x.
\end{equation}
And indeed, this is true for the set idiom!
Equation~\eqref{eq:comb-intro2} is very powerful if it holds for an idiom,
because it will allow us to permute opaque terms freely.

\todo

\subsection{Combinator Bases}\label{subsec:combinator-bases}

\begin{table}\centering
\begin{tabular}{cll}
Symbol & Reduction \\
\hline
$\mathbf{B}$ & $\mathbf{B} x y z = x (y z)$ \\
$\mathbf{I}$ & $\mathbf{I} x = x$ \\
\hline
$\mathbf{C}$ & $\mathbf{C} x y z = x z y$ \\
$\mathbf{K}$ & $\mathbf{K} x y = x$ \\
$\mathbf{W}$ & $\mathbf{W} x y = x y y$ \\
$\mathbf{S}$ & $\mathbf{S} x y z = x z (y z)$ \\
$\mathbf{H}$ & $\mathbf{H} x y z = x y (z y)$ \\
\end{tabular}
\caption{Useful combinators.}
\label{tab:combinators}
\end{table}

\begin{table}\centering
\begin{tabular}{ll}
Base & Example idioms \\
\hline
$\mathbf{BI}$ & state, list \\
$\mathbf{BIC}$ & set \\
$\mathbf{BIK}$ & \\
$\mathbf{BIW}$ & either \\
$\mathbf{BCK}$ & \\
$\mathbf{BKW}$ & \\
$\mathbf{BICW}$ & maybe \\
$\mathbf{BCKW}$ & stream, $\alpha \to$ \\
\end{tabular}
\caption{Substructures of BCKW.}
\label{tab:combinator-bases}
\end{table}

\todo
